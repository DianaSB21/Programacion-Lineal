\documentclass{article}

\usepackage[utf8]{inputenc}
\usepackage{amsmath}
\usepackage[spanish]{babel}
\title{Apuntes de programación lineal}

\author{Diana Sebastian Bernal}


\begin{document}


\maketitle


\section{Introducción}
\label{sec:introducción}

La forma estándar de un problema de programación lineal es: Dados una
matriz $A$ y vectores $b,c$, maximizar $c^Tx$ sujeto a $Ax\leq b$.
Si alguna variabe no esta acotada por cero, se hace un cambio de
variable.

\section{Ejemplos}
\label{sec:ejemplos}
Maximizar $x+2y$ sujeto a $x+y\leq 6$, $x\geq -1$,$y\geq 0$

\section{Tabla}
\label{sec:tabla}


\begin{tabular}{|c|c|c|}
  \hline
  &A&B\\
   \hline
   Máquina 1&1&2\\
   \hline
  Máquina 2&1 &2\\
\hline
  \end{tabular}

  
  
\section{Matrices}
\label{sec:matrices}
\begin{equation*}
  \label{eq:1}
 A= \begin{pmatrix}
    7&8&3\\
    2&0&1
  \end{pmatrix}
  
    B=
    \begin{pmatrix}
      0&3&9&1&-7\\
      2&-7&4&0&1\\
      1&3&5&2&-3
    \end{pmatrix}

  \end{equation*}

\end{equation}

                
\end{document}

