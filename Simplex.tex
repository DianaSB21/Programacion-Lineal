\documentclass{article}

\usepackage[spanish]{babel}
\usepackage{amsmath}
\usepackage[utf8]{inputenc}

\title{Método Simplex}
\author{Diana Sebastian Bernal}

\begin{document}

\maketitle

\section{Introducción}
\label{sec:introduccion}

El método simplex es un algoritmo para resolver problemas de
programacion lineal. Fue inventado por George Bernard Dantzing en el
año 1947.

\section{Ejemplo}
\label{sec:ejemplo}

Ilustraremos la aplicación del método simplex con un ejemplo.
\begin{equation*}
 \begin{aligned}
\text{Maximizar} \quad & 2x_{1}+2x_{2}\\
\text{sujeto a} \quad &
  \begin{aligned}
   2x_1+x_{2} &\leq 4\\
   x_{1}+2x_{2} &\geq 5\\
    x_{1},x_{2} &\geq 0
  \end{aligned}
\end{aligned}

\end{equation*}
Como una de las ecuciones tiene el simbolo $\geq$, entonces
multiplicamos a todos los miebros de la ecuacion por $-1$ para obtener
el problema en forma estandar.
\begin{equation*}
 \begin{aligned}
\text{Maximizar} \quad & 2x_{1}+2x_{2}\\
\text{sujeto a} \quad &
  \begin{aligned}
   2x_1+x_{2} &\leq 4\\
 -x_{1}-2x_{2} &\leq -5\\
    x_{1},x_{2} &\geq 0
  \end{aligned}
\end{aligned}
\end{equation*}

Para obtener la forma simplex, añadimos variables de holgura por cada
desigualdad.

\begin{equation*}
 \begin{aligned}
\text{Maximizar} \quad & 2x_{1}+2x_{2}\\
\text{sujeto a} \quad &
  \begin{aligned}
   2x_1+x_{2}+x_{3} &= 4\\
   -x_{1}-2x_{2}    +x_{4} &=-5\\
    x_{1},x_{2},x_{3},x_{4} &\geq 0
  \end{aligned}
\end{aligned}
\end{equation*}

A continuacion obtenemos \emph{tablero simplex} despejando las
variables de holgura.
\begin{equation*}
 \begin{aligned}
   x_3 &=4-2x_1-x_2\\
   x_4  &=-5+x_1+2x_1\\
   \hline
   z &=\phantom{-5}+2x_{1}+2x_{2}
  \end{aligned}

\end{equation*}

\end{document}

